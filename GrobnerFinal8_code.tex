\documentclass[12pt,a4paper]{article}
\usepackage{amsfonts,amsmath,amssymb,mathrsfs}
\usepackage{polynom}
\usepackage{amsthm}
\usepackage[utf8]{inputenc}


\newtheorem{theorem}{Theorem}
\newtheorem{lemma}{Lemma}
\theoremstyle{definition}
\newtheorem{definition}{Definition}
\newtheorem{remark}{Remark}

\title{
{\bf\Large INDIAN INSTITUTE OF TECHNOLOGY GANDHINAGAR}\\[1in]
{\bf\Large GR\"{O}BNER BASES OF RATIONAL NORMAL CURVES}\\[1in]
{\bf\large AMOGH PARAB (14110089)}\\[2mm]
{\bf\large KSHITEEJ JITESH SETH (14110068)}\\[1in]
{\bf\large SUPERVISOR: DR. INDRANATH SENGUPTA}\\[2mm]
}
 
\begin{document}
 \pagenumbering{gobble}
\maketitle
\newpage
\pagenumbering{arabic}


\section*{Abstract}

\noindent Let $n\geq 3$ be a natural number. Let $R = k[x_{0}, \ldots , x_{n}]$ be the 
polynomial ring in the indeterminates $x_{0}, x_{1} , \ldots , x_{n}$ over a field $k$. Let 
$A = \left[\begin{matrix}
x_{0} & x_{1} & \cdots & x_{n-1}\\
x_{1} & x_{2} & \cdots & x_{n}
\end{matrix}\right]$. 
Let $\mathcal{G}_n$ denote the set of all $2\times 2$ minors of the matrix $A$, i.e., 
$\mathcal{G}_n = \{x_{i}x_{j+1}-x_{i+1}x_{j} \mid 0 \leq i < j \leq n\}$. Let $I$ denote 
the  ideal  generated by $\mathcal{G}_n$ in $k[x_{0}, x_{1},...,x_{n}]$. Suppose that 
the monomial ordering in $R =  k[x_{0}, x_{1},...,x_{n}]$ is given by 
$x_{i_{0}} > x_{i_{1}}> \ldots > x_{i_{n}}$, with the lexicographic ordering of monomials in $R$, 
where $(i_{0}, i_{1}, \ldots , i_{n})$ denotes a permutation of the set $\{0, 1, \ldots , n\}$. 
We have classified all possible permutations $(i_{0}, i_{1}, \ldots , i_{n})$ of $\{0, 1, \ldots , n\}$ 
such that $\mathcal{G}_n$ is a Gr\"{o}bner basis of the ideal $I$.




\section*{Definitions}
\theoremstyle{definition}
\begin{definition}{Set $S_k \subset \mathbb{N}$:}

\noindent If monomial ordering in $R=k[x_0,x_1,\ldots , x_n]$ is given by $x_{i_{0}}> x_{i_{1}}> \ldots> x_{i_{k}}> \ldots> x_{i_{n}}$, then the set $S_k$ is defined as $S_{k}=\{i_{k}, i_{k+1}, i_{k+2}, \ldots, i_{n}\}$; $k=0,1,\ldots ,n$

\end{definition}

\paragraph*{Remark:}
$S_0$ is full set $\{0,1,\ldots, n\}$ and $S_n$ is singleton set $\{i_n\}$.

%%%%%%%%%%%%%%%Change%%%%%%%%%%%%%%%%%
\paragraph*{Example:}
For monomial ordering $x_2>x_0>x_1>x_4>x_3$ in $R=k[x_0, x_1, \ldots , x_4]$,\\$S_0 = \{2, 0, 1, 4, 3\}$, $S_1 = \{0, 1, 4, 3\}$, $S_2 = \{1, 4, 3\}$, $S_3 = \{4, 3\}$, $S_4 = \{3\}$.\\[1cm]
%%%%%%%%%%%%%%%%%%%%%%%%%%%%%%%

\begin{definition}{Property $P_{j}$ for given monomial order:}

\noindent If the monomial ordering in $R=k[x_0,x_1,\ldots , x_n]$ is given by $x_{i_{0}}> x_{i_{1}}> \ldots> x_{i_{j}}> \ldots> x_{i_{n}}$,where $0\leq j \leq n$  then the given monomial order is said to satisfy property $P_j$ if $i_{k}$ is either $max(S_{k})$ or $min(S_{k})\ \forall k\leq j$

\end{definition}

\paragraph*{Remark:}
If the given monomial order satisfies the property $P_j, \ \ 0< j \leq n$ then it satisfies property $P_k \ \ \forall k<j$. 

%%%%%%%%%%%%%%%change%%%%%%%%%%%%%%%
\paragraph*{Example:}
For monomial ordering $x_0>x_5>x_1>x_3>x_4>x_2$ in $R=k[x_0, x_1, \ldots , x_5]$, it satisfy properties $P_0$, $P_1$ and $P_2$; but NOT
$P_3$, $P_4$ and $P_5$.
%%%%%%%%%%%%%%%%%%%%%%%%%%%%%%%%%%


\section*{Theorem:}

Suppose that the monomial ordering in $k[x_{0} > x_{1} > \ldots > x_{n}]$ is given by $( n \geq 3)$ .
$x_{i_{0}} > x_{i_{1}} > \ldots > x_{i_{n}} $ with the lexicographic ordering. Let $\mathcal{G}_n$ denote the set of all $2\times 2$ minors of the matrix $A$, i.e., $\mathcal{G}_n = \{x_{i}x_{j+1}-x_{i+1}x_{j} \mid 0 \leq i < j \leq n\}$. Let $I$ denote the  ideal  generated by $\mathcal{G}_n$ in $k[x_{0}, x_{1},...,x_{n}]$. The set $\mathcal{G}_n$ is a Gr\"{o}ebner basis with respect to the said monomial order if and only if \\
given monomial order satisfies the property $P_{n-3}$.\\
That is $i_{k}$ is either min($S_{k}$) or max($S_{k}$) for $0 \leq k \leq n-3$\\.

And for $n=2\ G_n$ forms a Gr\"{o}bner basis.



\paragraph*{Remark:}
There is relaxation on properties $P_{n-2}, P_{n-1}$ and $P_{n}$. The monomial order may or may not satisfy the Properties $P_{n-2}, P_{n-1}$ and $P_{n}$.
















\section*{Proof for Only If part:}

The set $\mathcal{G}_n;\ \ n>2;$ is a Gr\"{o}ebner basis with respect to the said monomial order only if \\
given monomial order satisfies the property $P_{n-3}$.\\




\begin{theorem}
Suppose that the monomial ordering in $k[x_{0},x_{1}, \ldots , x_{n}]$ is given by $( n \geq 3)$ .
$x_{i_{0}} > x_{i_{1}} > \ldots > x_{i_{n}} $ with the lexicographic ordering. Let $\mathcal{G}_n$ denote the set of all $2\times 2$ minors of the matrix $A$, i.e., $\mathcal{G}_n = \{x_{i}x_{j+1}-x_{i+1}x_{j} \mid 0 \leq i < j \leq n\}$. Let $I$ denote the  ideal  generated by $\mathcal{G}_n$ in $k[x_{0}, x_{1},...,x_{n}]$. The set $\mathcal{G}_n$ a Groebner  basis with respect to the said monomial order only if $i_{0}$ is either 0 or n.
\end{theorem}

\begin{proof}
\noindent By method of contradiction.\\
\\
{\bf Case I: n=3}

$A = \left[\begin{matrix}
x_{0} & x_{1} & x_{2} \\
x_{1} & x_{2} & x_{3}\end{matrix}\right]$.\\
$\mathcal{G}_3$ = {$ x_{0}x_{2} - x_{1}^2, x_{0}x_{3} - x_{2}x_{1} , x_{1}x_{3} - x_{2}^2$} \\
Assume $i_{0}$  is  neither  0  or  3.\\
$\Rightarrow i_{0}$ = 1 or $i_{0}$ = 2 \\
\\
Subcase I] $i_{0} = 1$ i.e. $x_{1}$ largest\\
consider the $S$ polynomial \\
$S(x_{0}x_{3} - x_{2}x_{1} , x_{1}x_{3} - x_{2}^2) = x_{2}
^3 - x_{0}x_{3}^2 $ \\
$x_{2}^3 - x_{0}x_{3}^2$ does not tend to 0  \\
as LT of each polynomial in $\mathcal{G}_3$ contains $x_{1}$
which is not present in $x_{2}^3 - x_{0}x_{3}^2$ thus $\mathcal{G}_3$ does not form a Groebner Basis for $i_{0}$ = 1.\\
\\
Subcase II] $i_{0}$ = 2 i.e. $x_{2}$ is largest.\\

Consider the $S$ polynomial\\
$S(x_{0}x_{3} - x_{2}x_{1} , x_{0}x_{2} - x_{1}^2) = x_{1}^3 - x_{3}x_{0}^2 $ \\
$x_{2}^3 - x_{0}x_{3}^2$ does not tend to 0  \\
as LT of each polynomial in $\mathcal{G}_3$ contains $x_{2}$
which is not present in $x_{0}^2x_{3} - x_{1}^3$ thus $\mathcal{G}_3$ does not form a Groebner Basis for $i_{0}$ = 2.\\
\\
Thus $\mathcal{G}_3$ does not form a Groebner Basis if $i_{0}$ is neither 0 nor 3 which is a contradiction for n = 3\\
\\
\noindent
{\bf Case II: n = 4}
\\Assume $i_{0}$ is neither 0 nor 4
\\$\Rightarrow\  i_{0}= 1$ or $i_{0} = 2$ or $i_{0} = 3$ \\
$\mathcal{G}_3 = \{x_{0}x_{2} - x_{1}^2 , x_{0}x_{3} - x_{1}x_{2}, x_{0}x_{4} - x_{1}x_{3} , x_{1}x_{3} - x_{2}^2 , x_{1}x_{4} - x_{2}x_{3} ,  x_{2}x_{4} - x_{2}x_{4} - x_{3}^2\}$
\\
Subcase I] $i_{0}$ = 1 i.e. $x_{1}$ is largest.
\\
\\
Consider the $S$ polynomial \\
$S(x_{1}x_{3} - x_{2}^2 , x_{0}x_{4} - x_{1}x_{3}) $ = $x_{0}x_{4} - x_{2}^2$\\
which does not tend to 0 as except for $x_{2}x_{4} - x_{3}^2 $ all other polynomials LT contains $x_{1}$ which is not present in $x_{0}x_{4} - x_{2}^2 $ and $x_{2}x_{4} - x_{3}^2$ does not divide the S polynomial.
$\Rightarrow \mathcal{G}_3$ does not form a Groebner Basis for $i_{0}$ = 1
\\
\\
Subcase II] $i_{0}$ = 2 i.e. $x_{2}$ is largest.
\\
Consider the $S$ polynomial
$S(x_{0}x_{3} - x_{1}x_{2} , x_{1}x_{4} - x_{2}x_{3})$ = $x_{0}x_{3}^2 - x_{1}^2x_{4}$
which does not tend to 0 
as except for $x_{0}x_{4} - x_{1}x_{3}$ , all other polynomial's LT contain$ x_{2}$ which is not present in $x_{0}x_{4} - x_{1}x_{3}$.\\
Thus $\mathcal{G}_3$ does not form a Gr\"{o}ebner Basis for $i_{0}$ = 2.
\\
\\
Subcase III] $i_{0}$ = 3 i.e. $x_{3}$ is largest.
 \\
 Consider the $S$ polynomial
 $S(x_{3}x_{1} - x_{4}x_{0} , x_{3}x_{1} - x_{2}^2)$ = $x_{2}^2 - x_{4}x_{0}$
which do not tend to 0 
as except for $x_{0}x_{2} - x_{1}^2$ , all other polynomial's LT contain $x_{3}$ which is not present in $x_{2}^2 - x_{4}x_{0}$.\\
Thus $\mathcal{G}_3$ does not form a Gr\"{o}ebner Basis for $i_{0}$ = 3.
\\
\\
Thus $\mathcal{G}_3$ does not form a Gr\"{o}ebner Basis if $i_{0}$ is neither 0 nor 4 which is a contradiction. 
\\
\\
%%%%%%%%%%%%%%%%change%%%%%%%%%%%%%
\begin{lemma}
Let $R=k[x_0, x_1, \ldots, x_n]$ be polynomial ring $(n \geq 3)$. For the lexicographic ordering in indeterminants with property $x_{i_0} > x_{i_1} > \ldots > x_{i_n}$. Let $G_n$ be generator set defined by the set of all $2\times 2$ minors of  the matrix $A$. Let $I$ be the ideal generated by the generator $G_n$. Let $P = x_{i+1}x_{i-3}-x_{i-1}^2 \in R$ where $3 \leq i \leq n-1$. \\
If  $LT(x_{i}x_{i-2}-x_{i+1}x_{i-3}) = x_{i}x_{i-2}$ and $LT(x_{i}x_{i-2}-x_{i-1}^2) = x_{i}x_{i-2}$, then $P$ is not divisible by $G_n$. 
\end{lemma}

\begin{proof}
Possible divisors of $P$ from $G_n$ are $x_{i+1}x_{i-3}-x_{i}x_{i-2}$, $x_{i+1}x_{i-3}-x_{i+2}x_{i-4}$ (if $4 \leq i \leq n-2$) if $LT(P)=x_{i+1}x_{i-3}$ and $x_{i-1}^2-x_{i}x_{i-2}$ if $LT(P) = x_{i-1}^2$.\\
\\
Now there are two possibilities\\
\\
I. $LT(P)=-x_{i-1}^2$;\\ For this case possible divisor is $x_{i-1}^2-x_{i}x_{i-2}$. But we have $LT(x_{i-1}^2-x_{i}x_{i-2}) = x_{i}x_{i-2}$. Thus the leading term of $P$ is not divisible by the leading term of the divisor, hence $P$ is not divisible.\\
\\
II. $LT(P) = x_{i+1}x_{i-3}$;\\
Now there are two possible divisors $x_{i+1}x_{i-3}-x_{i}x_{i-2}$ and $x_{i+1}x_{i-3}-x_{i+2}x_{i-4}$ (if $4 \leq i \leq n-2$), thus two subcases depending upon which polynomial would divide the polynomial $P$ first.\\
\\
$x_{i+1}x_{i-3}-x_{i}x_{i-2}$ can not divide $P$ first because $LT(x_{i}x_{i-2}-x_{i+1}x_{i-3}) = x_{i}x_{i-2}$ and $LT(P) = x_{i+1}x_{i-3}$.\\
\\
If $x_{i+1}x_{i-3}-x_{i+2}x_{i-4}$ divides $P$ first ($4 \leq i \leq n-2$);\\  This implies that $LT(x_{i+1}x_{i-3}-x_{i+2}x_{i-4}) = x_{i+1}x_{i-3}$. After division we get\\
$P = x_{i+1}x_{i-3}-x_{i-1}^2 = 1\times (x_{i+1}x_{i-3}-x_{i+2}x_{i-4}) + (x_{i+2}x_{i-4}-x_{i-1}^2)$.\\
Thus the remainder $R_1$ after first step of division is $R_1=x_{i+2}x_{i-4}-x_{i-1}^2$.\\
Now again there are three possible divisors of $R_1$;\\ 1. $x_{i-1}^2-x_{i}x_{i-2}$\ \ \ \ \ \ \ \ \ \ \ \ \ \ \ \ \ \ \ \ \ \ \ \ \ \ \ \ \ \ \ \ \ \ \ \  if $LT(R_1)=-x_{i-1}^2$,\\2. $x_{i+2}x_{i-4}-x_{i+1}x_{i-3}$\ \ \ \  \ \ \ \ \  \ \ \ \ \ \ \ \ \ \ \ \  \ \ \  \ \  \ \ if $LT(R_1)=x_{i+2}x_{i-4}$.\\3. $ x_{i+2}x_{i-4}-x_{i+3}x_{i-5}(5 \leq i \leq n-3)$  \ \ \ \  \ \ \ if $LT(R_1)=x_{i+2}x_{i-4}$.  
\\
\\
We discard possibilities 1 and 2 because $LT(x_{i}x_{i-2}-x_{i-1}^2) = x_{i}x_{i-2}$ and $LT(x_{i+1}x_{i-3}-x_{i+2}x_{i-4}) = x_{i+1}x_{i-3}$.\\
After dividing $R_1$ by (3) we get;\\
$R_1 = x_{i+2}x_{i-4}-x_{i-1}^2 = 1\times (x_{i+2}x_{i-4}-x_{i+3}x_{i-5}) + (x_{i+3}x_{i-5}-x_{i-1}^2)$
\\
$\therefore R_2 = x_{i+3}x_{i-5}-x_{i-1}^2$.\\
\\
We claim that, 
At general step $m$, we get $R_m = x_{i+m+1}x_{i-m-3}-x_{i-1}^2$ ($m+3 \leq i \leq n-m-1$) with $LT(x_{i+m+1}x_{i-m-3}-x_{i+m}x_{i-m-2})=-x_{i+m}x_{i-m-2}$    % ($\because R_{m-1}$ is divisible by $x_{i+m+1}x_{i-m-3}-x_{i+m}x_{i-m-2}$, thus leading terms should match).\\
\\
\\
Claim is true for the case $m=1$ ($\because R_1=x_{i+2}x_{i-4}-x_{i-1}^2$ and $LT(x_{i+2}x_{i-4}-x_{i+1}x_{i-3}) = -x_{i+1}x_{i-3}$).
\\
\\
Suppose the claim is true for the case $m=p$.\\
$\therefore R_p = x_{i+p+1}x_{i-p-3}-x_{i-1}^2$.\\
Now possible diviors are $x_{i-1}^2-x_{i}x_{i-2}$ if $LT(R_p)= x_{i-1}^2$ and $ x_{i+p+1}x_{i-p-3}- x_{i+p}x_{i-p-2}$, $x_{i+p+1}x_{i-p-3}-x_{i+p+2}x_{i-p-4}$ ($p+4 \leq i \leq n-p-2$) if $LT(R_p) = x_{i+p+1}x_{i-p-3}$.\\
\\
We discard the possibilities 1 and 2 because $LT(x_{i-1}^2-x_{i}x_{i-2}) = -x_{i}x_{i-2}$ and $LT( x_{i+p+1}x_{i-p-3}- x_{i+p}x_{i-p-2}) = - x_{i+p}x_{i-p-2}$.\\
\\
So, after dividing $R_p = x_{i+p+1}x_{i-p-3}-x_{i-1}^2$ by $x_{i+p+1}x_{i-p-3}-x_{i+p+2}x_{i-p-4}$ (assuming $LT(x_{i+p+1}x_{i-p-3}-x_{i+p+2}x_{i-p-4}) = x_{i+p+1}x_{i-p-3}$), we get remainder as $x_{i+p+2}x_{i-p-4}-x_{i-1}^2$, which can be rewritten as $R_{p+1} = x_{i+(p+1)+1}x_{i-(p+1)-3}-x_{i-1}^2$.\\ Here we assumed $LT(x_{i+p+2}x_{i-p-4}-x_{i+p+1}x_{i-p-3}) = LT(x_{i+(p+1)+1}x_{i-(p+1)-3}-x_{i+(p+1)}x_{i-(p+1)-2} = -x_{i+(p+1)}x_{i-(p+1)-2})$\\
\\
Thus the claim is true for $m=p+1$ Therefor by principle of mathematical induction we can say that the claim is true.\\
\\
So after some repeatations the process will terminate at such $m$ where either $i+m+1 = n$ or $i-p-3 = 0$. Then such $R_m$ will look like $x_{0}x_{q}-x_{i-1}^2$ or $x_{q}x_{n} - x_{i-1}^2$ which is not further divisible by any of the polynomial from $G_n$ ($\because$ above claim).\\
\\ This proves our lemma.
\end{proof}

%%%%%%%%%%%%%%%%%%%%%%%%%%%%%%%%%%%%%%%


{\noindent \bf Case III: n $\geq$ 5} 
\\Assume $i_{0}$ is neither 0 nor n \\
Let $i_{0}$ = i i.e. $x_{i}$ is largest.
such that $0 < i < n$\\
$\Rightarrow$ either $i-3 \geq 0 $ or $ i + 3 \leq n$ \\
for if $i-3 < 0 $
\\
$\Rightarrow i+3 < 6$\\
$\Rightarrow i+3 \leq 5 \leq n \ldots $ as i is an  
integer.
\\
\\
Subcase I] $i - 3 \geq 0$
\\
Consider the $S$ polynomial \\
$S(x_{i}x_{i-2} - x_{i+1}x_{i-3} , x_{i}x_{i-2} - x_{i-1}^2) = x_{i-1}^2 - x_{i+1}x_{i-3}$
\\
Only possible divisors are $x_{i-1}^2-x_{i}x_{i-2}, x_{i+1}x_{i-3}-x_{i}x_{i-2}$ and $x_{i+1}x_{i-3}-x_{i+2}x_{i-4}$.\\
In this $x_{i-1}^2-x_{i}x_{i-2}$ and $x_{i+1}x_{i-3}-x_{i}x_{i-2}$ will not divide the S-Polynomial as the leading term of S-Polynomial is not divisible by the leading terms of $2\times 2$ minor.\\
Whereas $x_{i+1}x_{i-3}-x_{i+2}x_{i-4}$ gives nonzero remainder after division from lemma 1.
\\
Thus S-Polynomial does not tend to 0 on division by $\mathcal{G}$. Thus $\mathcal{G}$ does not form a Gr\"{o}ebner Basis.
\\
\\
Subcase II] $i + 3 \leq n$
\\
Consider the $S$ polynomial\\
$S(x_{i}x_{i-2} - x_{i+1}x_{i-3} , x_{i}x_{i-2} - x_{i-1}^2) = x_{i-1}^2 - x_{i+1}x_{i-3}$
\\
By same reasons as above S-Polynomial 
does not tend to 0 on division by $\mathcal{G}$. Thus $\mathcal{G}$ does not form a Gr\"{o}oebner Basis.
\\
Thus $\mathcal{G}$ does not form a gr\"{o}oebner basis for $n \geq 5$.\\
Thus $\mathcal{G}$ does not form a gr\"{o}ebner basis for any $n \geq 3$,\\
if $i_{0}$ is neither 0 nor n.\\
Hence the contradiction.\\
$\Rightarrow i_{0} = 0$ or $i_{0} = n$.\\
\end{proof}

\begin{lemma}
If monomial ordering is $x_{i_{0}}> \ldots> x_{i_{n}}$, with property $P_{j-1}$; $1\leq j \leq n$ and if $min(S_j) \leq m \leq max_(S_j)$; then $m \in S_{j}$
\\
That is all the integers in between $min(S_{j})$ and $max(S_{j})$ are contained in $S_{j}$, i.e. $S_{j}$ is of the form \{i, i+1, \ldots i+k\}.

\end{lemma}


\begin{proof}
Assume $m\notin S_{j}$\\
then $x_{m} > x_{l} \ \  \forall \ l \in S_{j}  \ldots$ if $x_{m} < x_{l}$ for some $l \in S_{j}$ then $m\in S_{j}$ by definition.\\
$\Rightarrow \ m=i_{p}$ for some $p<j \ldots$ as $i_{p} \in S_j$ for $p \geq j$ 
\\
$\Rightarrow \ \ m=min(S_{p})$ or $m=max(S_{p})$\\
From definition of the set $S_{k}$ we know that $S_{j} \subset S_{p}$
but $m \geq min(S_{j}) \in S_{j} \subset S_{p}$
\\
thus $m$ can't be $min(S_{p})$\\
Similarly $m \leq max(S_{j}) \in S_{j} \subset S_{p}$
\\
$\therefore m\neq max(S_{p})$\\
Which is a contradiction.

\end{proof}




\begin{theorem}
\noindent Suppose that the monomial ordering in $k[x_{0} > x_{1} > \ldots > x_{n}]$ is given by $( n \geq 3)$ .
$x_{i_{0}} > x_{i_{1}} > \ldots > x_{i_{n}} $ with the lexicographic ordering. Let $\mathcal{G}_n$ denote the set of all $2\times 2$ minors of the matrix $A$, i.e., $\mathcal{G}_n = \{x_{i}x_{j+1}-x_{i+1}x_{j} \mid 0 \leq i < j \leq n\}$. Let $I$ denote the  ideal  generated by $\mathcal{G}_n$ in $k[x_{0}, x_{1},...,x_{n}]$. The set $\mathcal{G}_n$ a Gr\"{o}ebner  basis with respect to the said monomial order only if \\
$i_{k}$ is either min($S_{k}$) or max($S_{k}$) \\
for $0 \leq k \leq n-3$\\
that is given monomial order satisfies the property $P_{n-3}$
\end{theorem}
\paragraph*{Remark:} Monomial ordering need not satisfy the property $P_{n-2}, P_{n-1}$ or $P_{n}$


\begin{proof}
\noindent\\
Using the method of induction on subscript number of the property $P_{k}$\\
True for $k = 0$ from Theorem I\\
Consider true for $k = j-1; 1\leq j \leq n-3$. i.e property $P_{j-1}$ is satisfied and we have to show that property $P_{j}$ is also satisfied by the monomial ordering. 
\\
Assume not true for $k = j \leq n-3$
\\
$\therefore \ min(S_{j}) < i_{j} < max(S_{j})$\\
Now, $j \leq n-3 \Rightarrow$ cardinality of $S_j$ is at least 4\\
\\



{\noindent \bf Case I: $|S_j| = 4$}
\\
From induction hypothesis, property $P_{j-1}$ is satisfied and from lemma 2 we can say that $S_{j}$ is of the form $\{i, i+1, i+2 , i+3\}$.\\
Because $min(S_{j}) < i_{j} < max(S_{j})$there are only two possibilities of $S_{j}$ as follows.\\
$S_{j}=\{i_{j}-1, i_{j}, i_{j}+1, i_{j}+2\}$ or $S_{j}=\{i_{j}-2, i_{j-1}, i_{j}, i_{j}+1\}$\\

\noindent Now, consider $S_{j}=\{i_{j}-1, i_{j}, i_{j}+1, i_{j}+2\}$
then consider\\ 
$S(x_{i_{j}-1}x_{i_{j}+2}-x_{i_{j}}x_{i_{j}+1}, x_{i_{j}}x_{i_{j}+2} - x_{i_{j}+1}^{2}) = x_{i_{j}-1}x_{i_{j}+2}^{2}-x_{i_{j}+1}^{3}$\\
\\
if $LT(x_{i_{j}-1}x_{i_{j}+2}^{2}-x_{i_{j}+1}^{3})=x_{i_{j}+1}^3$
\\
then $S\nrightarrow 0$ as only divisor to $x_{i_{j}+1}^3$ is $x_{i_{j}+1}^2-x_{i_{j}+2}x_{i_{j}}$ who's leading term is $x_{i_{j}+2}x_{i_{j}}$\\
\\

\noindent if $LT(x_{i_{j}-1}x_{i_{j}+2}^{2}-x_{i_{j}+1}^{3})=x_{i_{j}-1}x_{i_{j}+2}^2$    
\\
Then only possible divisors are  $x_{i_{j}+2}^2-x_{i_{j}+1}x_{i_{j}+3}, x_{i_{j}-1}x_{i_{j}+2}-x_{i_{j}}x_{i_{j}+1}$ and $x_{i_{j}-1}x_{i_{j}+2}-x_{i_{j}-2}x_{i_{j}+3}$(if exists)\\
In the case of $x_{i_{j}-1}x_{i_{j}+2}-x_{i_{j}}x_{i_{j}+1}$, $LT(x_{i_{j}-1}x_{i_{j}+2}-x_{i_{j}}x_{i_{j}+1}) = -x_{i_{j}}x_{i_{j+1}}$ which does not divide $x_{i_{j}-1}x_{i_{j}+2}$.\\
In the case of $x_{i_{j}-1}x_{i_{j}+2}-x_{i_{j}-2}x_{i_{j}+3}$; \\
$LT(x_{i_{j}-1}x_{i_{j}+2}-x_{i_{j}-2}x_{i_{j}+3}=x_{i_{j}-2}x_{i_{j}+3}$ as $x_{i_{j}+3}>x_{i_{j}-1}, x_{i_{j}+2}$\\
and in the case of $x_{i_{j}+2}^2-x_{i_{j}+1}x_{i_{j}+3}$; $LT(x_{i_{j}+2}^2-x_{i_{j}+1}x_{i_{j}+3})=x_{i_{j}+1}x_{i_{j}+3}$
\\
\\
Similar arguments goes for $S_{j}=\{i_{j}-2, i_{j}-1, i_{j}, i_{j}+1\}$
\\ Thus for cardinality of $S_{j}=4$,  $G_{n}$ does not form a Gr\"{o}bner Basis.
Hence contradiction.
\\
\\
{\noindent \bf case II: $|S_j|=5$}
\\
From assumption and lemma 2 only possible cases are, \\
$S(j)={i_j-1,i_j,i_j+1,i_j+2,i_j+3}$\\
$S(j)={i_j-2,i_j-1,i_j,i_j+1,i_j+2}$\\
$S(j)={i_j-3,i_j-2,i_j-1,i_j,i_j+1}$\\
\\
Consider following examples in each cases respectively. \\
$S(f_{i_{j}-1,i_{j}+2},f_{i_{j},i_{j}+1})=x_{i_{j}-1}x_{i_{j+3}}-x_{i_{j}+1}^{2}$\\
$S(f_{i_{j}-2,i_{j}},f_{i_{j}-1,i_{j}+1})=x_{i_{j}-2}x_{i_{j}+1}^{2}-x_{i_{j}-1}^{2}$\\
$S(f_{i_{j}-1,i_{j}+2},f_{i_{j},i_{j}+1})=x_{i_{j}-1}x_{i_{j}+3}-x_{i_{j}+1}^2$\\
are counter examples to each case respectively. Arguments for example 1 and 3 are similar as in case I.\\
For case 2\\
If LT is $x_{i_{j}-1}^{2}$, then LT of only possible divisor i.e. $LT(x_{i_{j}-1}^{2}-x_{i_{j}}x_{i_{j}-2})=x_{i_{j}}x_{i_{j}-2}$ for the reason $x_{i_{j}}>x_{i_{j}-2}$. Thus does not divide.\\
And is LT is $x_{i_{j}-2}x_{i_{j}+1}^{2}$; then from lemma 1 we can say that $G_{n}$ does not divide.
\\ Thus for cardinality of $S_{j}=5$,  $G_{n}$ does not form a Gr\"{o}bner Basis.
Hence contradiction.
\\
\\
\\
{\bf \noindent Case III: $|S_{j}|\geq 6$}\\
From assumption and lemma 1.1 we can say that\\
either $\{i_j-1, i_j,i_j+1,i_j+2,i_j+3\} \in S_{j}$ \\
or $\{i_j-3,i_j-2,i_j-1,i_j,i_j+1\} \in S_{j}$ 
\\
\\
\noindent Consider the case where,\\ $\{i_j-3,i_j-2,i_j-1,i_j,i_j+1\} \in S_{j}$\\

\noindent Consider,\\
$S(f_{i_{j},i_{j}-3}, f_{i_{j}-2,i_{j}-1})=x_{i_{j}-1}^2-x_{i_{j}+1}x_{i_{j}-3}$\\
if $LT(S)=x_{i-1}^2$ then only possible divisor is $x^{2}_{i_{j}-1}-x_{i_{j}}x_{i_{j}-2}$ who's leading term is $x_{i_{j}}x_{i_{j}-2}$\\
\\
if $LT(S)=x_{i_{j}+1}x_{i_{j}-3}$ then possible divisors are $x_{i_{j}+1}x_{i_{j}-3}-x_{i_{j}}x_{i_{j}-2}$ and $x_{i_{j}+1}x_{i_{j}-3}-x_{i_{j}+2}x_{i_{j}-4}$\\
The first one is not possible as leading term is $x_{i_{j}}x_{i_{j}-2}$.
For second possibility; from lemma 1 we can say that remainder after division by $G_{n}$ is non zero.\\
\\
Similar arguments will go for other possibility of the set $S_{j}$.\\
\\Thus for cardinality of $S_{j}\geq 6$,  $G_{n}$ does not form a Gr\"{o}bner Basis.
Hence contradiction.\\
\\

So, the assumption we made was wrong \\
$\therefore$ the set "$G_{n}$" forms a Gr\"{o}bner basis only if monomial order satisfies the property $P_{n-3}$. i.e.$i_{j}$ is either $max(S_{j})$ or $min(S_{j}); \forall i_{j}\leq n-3$.
\end{proof}

\section*{Proof for If part:}
\theoremstyle{definition}

%%%%%%%%%%%%%change%%%%%%%%%%%
\begin{definition}{Mapping $\phi$}
Suppose that the monomial ordering in $R_{n+1}=k[x_0,x_1,\ldots ,x_{n+1}]$ is given by $x_{i_{0}}>x_{i_{1}}> \ldots >x_{i_{n+1}}$. Consider set $A_{n+1}=\{x_0, x_1, \ldots ,x_{n+1}\}$ of all indeterminants in $R_{n+1}$  and the set $A_{n}=\{x_0, x_1, \ldots ,x_n \}$ of all indeterminants in $R_{n}$.\\
Let $x_{i_{a}} \in A_{n+1}$
then mapping $\phi$ is defined as,\\
$\phi:A_{n+1}/ \{x_{i_{a}}\} \rightarrow A_{n}$\\
$\phi (x_{i})=x_{i}$ \ \ \ \ \ \ if $i<i_{a}$\\
$\phi(x_{i})=x_{i-1}$ \ \ \ if $i>i_{a}$\\
It is easy to show that this is a one-to-one and onto map.\\
Moreover, we extend the definition to all polynomials not containing $i_{a}$ as,\\
$\phi(Ax^\alpha + Bx^\beta)=A\phi(x^\alpha)+B\phi(x^\beta)$ \ \ where $\alpha_{i_a}=\beta_{i_a}=0$\\
$\phi (Af+Bg)= A\phi(f)+B\phi(g)$\ \ \ $f,g \in k[x_0, \ldots , x_n]$\\
$\phi(x_{0}^{\alpha_{0}}x_1^{\alpha_{1}} \ldots x_{n+1}^{\alpha_{n+1}})=\phi(x_0)^{\alpha_0}\phi(x_1)^{\alpha_1} \ldots \phi(x_{n+1})^{\alpha_{n+1}}$ \ \ where $\alpha_{i_a}=0$\\
This is also a one to one and onto map.
\end{definition}

\paragraph*{Example:}
For monomial ordering $x_0>x_5>x_1>x_3>x_4>x_2$ in $R=k[x_0, x_1, \ldots , x_5]$, mapping $\phi$ from $A_{5}/\{x_1\}=\{x_0, x_2, x_3, x_4, x_5\} \rightarrow A_{4}=\{x_0, x_1, x_2, x_3, x_4\}$ is given by \\
$\phi = \{(x_0, x_0), (x_5, x_4), (x_3, x_2), (x_4, x_3), (x_2, x_1)\}$
%%%%%%%%%%%%%%%%%%%%%%%%%%%%%%%%%%%%%5

\begin{definition}{Mapping $\phi$ on the order}
Suppose that the monomial ordering $>_{n+1}$ in $R_{n+1}=k[x_0,x_1,\ldots ,x_{n+1}]$ is given by $x_{i_{0}}>x_{i_{1}}> \ldots >x_{i_{n+1}}$. Then the monomial ordering $\phi(>_{n+1})$ in $R_{n}=k[x_0,x_1,\ldots ,x_{n}]$ is defined by $\phi(x_{i_{0}})>\phi(x_{i_{1}})> \ldots >\phi(x_{i_{n+1}})$ where nothing maps at the pace of $i_{a}$.  

\end{definition}
%%%%%%%%%%%%%%5change%%%%%%%%%%%%%%5
\paragraph*{Example:}
For monomial ordering $>_5=x_0>x_5>x_1>x_3>x_4>x_2$ in $R=k[x_0, x_1, \ldots , x_5]$, mapping $\phi$ for invarient indeterminant $x_1$ maps $>_5$ to $>_4=x_0>x_4>x_2>x_3>x_1$
%%%%%%%%%%%%%%%%%%%%%%%%%%%%%%%%%%%%%%%%%%

\begin{lemma}
Let $f=x_{i}x_{j+1}-x_{i+1}x_{j}$ be $2\times 2$ minor of from $G_{n+1}$ such that $f$ does not contain $x_{i_{a}}$, then $\phi(f)$ is also a $2 \times 2$ minor from $G_{n}$. 
\end{lemma}
\begin{proof}
Consider $x_{i}x_{j+1}-x_{j}x_{i+1}$ which is a $2\times 2$ minor and neither of i, i+1, j, j+1 is $i_{a}$. It is enough to show that if $x_{i}$ maps to $x_{i'}$ then $x_{i+1}$ maps to $x_{i'+1}$.
\\

\paragraph*{Case 1}
$i<i_{a}$ \\
$\Rightarrow$ $i+1<i_{a}$\\
Thus i maps to i and i+1 maps to i+1
\paragraph*{Case 2}
$i>i_{a}$ \\
$\Rightarrow$ $i+1>i_{a}$\\
Thus i maps to i-1 and i+1 maps to i\\
\paragraph*{}
Thus if $x_{i}$ maps to $x_{i'}$ then $x_{i+1}$ maps to $x_{i'}+1$, which proves that polynomials are nothing but $2\times 2$ minor of $2\times n$ Matrix.
\end{proof}

\begin{lemma}
Suppose that $<_1$ and $<_2$ denote the monomial orders of $k[x_0, x_1, \ldots, x_{n+1}]$ and $k[x_0, x_1, \ldots , x_n]$ respectively, such that $\phi(<_{1})=<_{2}$. If $0 \neq f \in k[x_0, \ldots , x_{n+1}]$ and $x_{i_a}$ does not occur in $f$ then,\\
$\phi(LT_{<_{1}}(f))=LT_{<_2}(\phi(f))$\\
\end{lemma}
\begin{proof}
Let $<_1=(x_{i_{0}}>x_{i_{1}}>\ldots >x_{i_{n+1}})$. \\
Let $x=x_{i_{0}}x_{i_{1}}\ldots x_{i_{n+1}}$.\\ 
Let $\alpha = (\alpha_{0}\alpha_{1}\ldots \alpha_{n+1})$ such that $\alpha_{a}=0$.
Let $LT_{<_{1}}(f)=x^{\alpha}$.\\
Let $x^{\alpha_{1}}$ be any arbitrary term in $f$ other than $x^\alpha$ \\
Let $i^{th}$ entry of $\alpha-\alpha_{1}$ be non zero.\\
As $x^{\alpha}=LT_{<_{1}}(f)$, $\alpha(i)-\alpha_{1}(i)>0$.\\
We have $<_2=(\phi(x_{i_{0}})>\phi(x_{i_{1}})>\ldots >\phi(x_{i_{n+1}}))$\\
and $\phi(x)=\phi(x_{i_{0}})\phi(x_{i_{1}})\ldots \phi(x_{i_{n+1}})$.\\
Then from definition 3 we have, $\phi(x^{\alpha})=\phi(x)^{\alpha}$ which is a term of $\phi(f)$. Similarly $\phi(x^{\alpha_{1}})=\phi(x)^{\alpha_{1}}$ is also a term of $\phi(f)$.\\
Now, as first non-negative entry of $\alpha-\alpha_{1}$ is positive and as $\alpha$ hence $\alpha_1$ is arbitrary $\phi(x^{\alpha})$ is leading term of $\phi(f)$ with respect to monomial order $<_2$.\\
Hence, $\phi(LT_{<_{1}}(f))=LT_{<_2}(\phi(f))$ is proved.
  
\end{proof}


\begin{theorem}
Suppose that the monomial ordering in $k[x_{0},x_{1}, \ldots , x_{n}]$ is given by $( n \geq 3)$ .
$x_{i_{0}} > x_{i_{1}} > \ldots > x_{i_{n}} $ with the lexicographic ordering. Let $\mathcal{G}_n$ denote the set of all $2\times 2$ minors of the matrix $A$, i.e., $\mathcal{G}_n = \{x_{i}x_{j+1}-x_{i+1}x_{j} \mid 0 \leq i < j \leq n\}$. Let $I$ denote the  ideal  generated by $\mathcal{G}_n$ in $k[x_{0}, x_{1},...,x_{n}]$. The set $\mathcal{G}_n$ a Gr\"{o}bner  basis with respect to the said monomial order if given monomial  order satisfy the property $P_{n-3}$
\end{theorem}

\begin{proof}
The proof will follow the method of induction over the number of variables "$N$".\\
The statement is trivial in the case $N=2$ but we will go one step ahead and show that using a computer program (appended below) that the statement is also true for $N=2,\ldots , 7$.\\
Lets assume the statement is True for $N=n$, then we have to show that the statement is also True for $N=n+1$.\\
\\
Now, consider the S-Polynomial of $2\times 2$ Minors $f=x_{i}x_{j+1}-x_{j}x_{i+1}$ and $g=x_{l}x_{m+1}-x_{m}x_{l+1}$ as $S\left(f,g \right) $\\ 
As $n>7$, there exists a $x_{i_{a}}$ such that $x_{i_{a}}$ does not occur in any one of the 4 monomials appearing in $f$ and $g$, for the reason that there can be as the most 8 distinct variables that may occur in 4 monomials.  \\
\\

Now consider $x_{i_{a}}$ is not present in given pair of S-polynomial where monomial ordering is $<_1=x_{i_{0}} > \ldots > x_{i_{a-1}} > x_{i_{a}} > x_{i_{a+1}} > \ldots > x_{i_{n+1}}$.\\
Here we can apply mapping $\phi$ as defined in the definition 3.\\
Lemma 2 tells that $\phi(f)$ and $\phi(g)$ are both $2 \times 2$ minors from set $G_{n}$. \\
As $\phi(f)$ and $\phi(g)$ are both $2 \times 2$ minors from set $G_{n}$, using Induction Hypothesis we can say that the S-Polynomial of $\phi(f)$ and $\phi(g)$
is divisible by $G_{n}$. More precisely \\
\\
$S(\phi(f), \phi(g))=\sum\limits_{i} a_{i,j',k'}f_{j'}g_{k'}\ \ \ \ a_{i,j',k'} \in k[x_{0}, \ldots, x_{n}]$.
\\
\\
Division algorithm tells that\\
$multideg(a_{i'j'k'}) \leq multideg(S(\phi(f),\phi(g)))$\\
\\
As $\phi$ being a one to one and onto map from $A_{n+1}/\{i_{a}\}$ to $A_{n}$. We can define $\phi^{-1}$.\\
Applying $\phi^{-1}$ to above equation, we get\\
$\phi^{-1}(S(\phi(f), \phi(g)))=\phi^{-1}(\sum\limits_{i}a_{i,j',k'}f_{j'}g_{k'})$
\\
\\
But $\phi^{-1}(S(\phi(f), \phi(g)))$ is nothing but $S(f, g)$\\
%need to explain???
and $\phi^{-1}(\sum\limits_{i}a_{i,j',k'}f_{j'}g_{k'})$ is nothing but $\sum\limits_{i}a'_{i,j,k}f_{j}g_{k}$ where $f_{j}$ and $g_{k}$ are both $2\times 2$ minors in $R_{n+1}$\\
\\
Applying lemma 3 to above in-equation we will get\\
$multideg(a'_{i,j,k}) \leq multideg(S(f,g))$
\\
This shows that $S(f,g)$ is divisible by $G_{n+1}$.\\
Hence $G_{n+1}$ is also a Gr\"{o}bner basis.\\
This completes the proof




\end{proof}



\section*{Result}
Using Theorem 2 and Theorem 3 we can state the following.\\
\\

\noindent Suppose that the monomial ordering in $k[x_{0} > x_{1} > \ldots > x_{n}]$ is given by $( n \geq 3)$ .
$x_{i_{0}} > x_{i_{1}} > \ldots > x_{i_{n}} $ with the lexicographic ordering. Let $\mathcal{G}_n$ denote the set of all $2\times 2$ minors of the matrix $A$, i.e., $\mathcal{G}_n = \{x_{i}x_{j+1}-x_{i+1}x_{j} \mid 0 \leq i < j \leq n\}$. Let $I$ denote the  ideal  generated by $\mathcal{G}_n$ in $k[x_{0}, x_{1},...,x_{n}]$. The set $\mathcal{G}_n$ a Gr\"{o}ebner  basis with respect to the said monomial order if and only if \\
$i_{k}$ is either min($S_{k}$) or max($S_{k}$) \\
for $0 \geq k \geq n-3$\\
that is given monomial order satisfies the property $P_{n-3}$

%%%%%%%%%%%%%%%%%change%%%%%%%%%%%%%%%%
\section*{Examples}

Let $R=k[x_0, x_1, \ldots, x_5]$ be the polynomial ring over field k. Let $G = \{x_ix_{j+1}-x_{i+1}x_j\ |\ 0\leq i<j \leq 5\}$. Let $I$ denote the ideal generated by $G$. Then following are the some examples of lexicographic ordering where G forms a Grobner basis for ideal I.

\begin{enumerate}
\item For monomial ordering $x_0>x_1>x_2>x_3>x_4>x_5$ in $R=k[x_0, x_1, \ldots , x_5]$; generator $\mathcal{G}_5$ {\bf is} a Gr\"{o}ebner basis.
\item For monomial ordering $x_0>x_5>x_1>x_3>x_4>x_2$ in $R=k[x_0, x_1, \ldots , x_5]$; generator $\mathcal{G}_5$ {\bf is} a Gr\"{o}ebner basis.
\item For monomial ordering $x_5>x_2>x_1>x_0>x_4>x_3$ in $R=k[x_0, x_1, \ldots , x_5]$; generator $\mathcal{G}_5$ {\bf does not form} a Gr\"{o}ebner basis.
\item For monomial ordering $x_0 > x_5 > x_3 > x_1 > x_2 > x_4$ in $R=k[x_0, x_1, \ldots , x_5]$; generator $\mathcal{G}_5$ {\bf does not form} a Gr\"{o}ebner basis.

\end{enumerate}
%%%%%%%%%%%%%%%%change%%%%%%%%%%%%%%%%%%


%%%%%%%%%%%%%%%%change%%%%%%%%%%%%%%
\section*{Appendix:}
\subsection*{A}
Following is the pseudocode of our program.\\[2cm]
\begin{verbatim}
########################## GB OF RNC #########################

def Minor_Gen(n):             #Gives Generator Set G 
    minor = []
    if n==2:
        return [(x_0x_2 - x_1^2)]
    else:
        i = 1
        while i < 1:
            minor.append([x_(i-1)x_n - x_(n-1)x_i])
            i = i+1
    minor = minor + Minor_Gen(n-1)
    return minor


def S_Poly_Gen(n):            #Gives all S_Polynomials of Generator G  
    minor = Minor_Gen(n)
    S_List = []
    if n==2:
        return S_Poly(minor[1], minor[2])
    else:
        i = 1
        while i < n:
            S_List.append(S_Poly(minor[i], minor[n]))
            i = i+1
    S_List = S_List + S_Poly_Gen(n-1)
    return S_List


def Check(n):                 #Checks if Generator is Groebner basis 
    minor = Minor_Gen(n)      #Uses Buchberger’s Criterion 
    S_list = S_Poly_Gen(n)
    for poly in S_List:
        if minor divides poly:
            pass
        else:
            return False
    return True
\end{verbatim}
%%%%%%%%%%%%%%%%change%%%%%%%%%%%%%%%%

\subsection*{B}
Following is the Python code we used to check our result.\\

\noindent Use "isgrobner(n, order)", for a specific permutation order, where order is of the form $[i_0,i_1, \ldots ,i_n]$ for the monomial order $x_0>x_1> \ldots > x_n$ 
\\
Program consists of two parts. First, where we define a polynomial to Python and its operations like addition, multiplication, division. This part also gives LCM and S-polynomial of given polynomials.\\
Second part deals with only Rational Normal Curve. It gives the set $G_{n}$, i.e. set of all $2\times 2$ minors of A. Calculates its all s-polynomials. 
\\
For incorporating given order to polynomials we just changed the positions of monomials with respect to given order. Eg. $x_{0}^2x_{1}x_{3}^7$ with monomial ordering $x_{1}>x_{3}>x_{2}>x_{0}$, is same as $x_{0}x_{1^7}x_{3}^2$ with monomial ordering $x_{0}>x_{1}>x_{2}>x_{3}$.\\
We used Buchberger's criterion to determine if given generator with given monomial order is a Grobner basis or not. 
   

\begin{verbatim}
################GrobnerBasisOfRationalNormalCurve #####################
from numpy import *
from copy import deepcopy

"""Following is the program to check if a given generator
is a Grobener basis or not for given monomial order."""


def array_to_object(arrayin):
    obj=[]
    for i in list(arrayin):
        obj.append(tuple(i))
    return obj


class Poly(object):                                  #constructing multivariable polynomial structure
    """takes list of tuples
        each list represent a monomial with first entry as coeficient
        5xy-2x^2 = Poly([(5,1,1),(-2,2,0)])"""
    def __init__(self, poly):
        self.vari=deepcopy(len(poly[0])-1)
        zeropoly=[0]
        zeroterm=(0,)*(self.vari+1)
        zeropoly[0]=zeroterm
        self.zero=zeropoly
        dtype=[("0", float)]
        for i in range(self.vari):
            x=("%d"%(i+1), int)
            dtype.append(x)
        if len(poly)==0:
            self.poly=[]
            return None
        #self.poly=poly 
        d={}
        for i in poly:
            if i[1:] not in d.keys():
                if float(i[0]) != 0.0:
                    d[i[1:]]=i
            else:
                if d[i[1:]][0]+i[0]==0.0:
                    del d[i[1:]]
                else:
                    dummylist=list(d[i[1:]])
                    dummylist[0]=dummylist[0]+i[0]
                    d[i[1:]]=tuple(dummylist)
        unsortpoly=d.values()
        unsortarray=array(unsortpoly, dtype=dtype)
        if len(unsortpoly)==0:
            self.poly=self.zero
            return None
        order=[]
        for i in range(self.vari):
            x="%d"%(i+1)
            order.append(x)
        revsort=sort(unsortarray, order=order)
        reqlist=list(revsort)
        reqlist.reverse()
        self.poly=array_to_object(reqlist)

    def __eq__(self, other):
        return self.poly == other.poly

    def __ne__(self, other):
        return self.poly != other.poly

    def LT(self):                     #note o/p is tuple 
        return self.poly[0]

    def leadingterm(self):                          #gives leading term
        leadingterm=[0]
        leadingterm[0]=self.LT()
        return Poly(leadingterm)

    def multideg(self):
        return self.LT()[1:]

    def isdivisible(self, other):        
    """checks if the leading term of the polynomial is divisible by 
    the leading term of the other polynomial."""
        for i in range(len(self.multideg())):
            if self.multideg()[i]<other.multideg()[i]:
                return False
        return True

    def __add__(self, other):    #adds
        added = self.poly+other.poly
        return Poly(added)

    def __sub__(self, other):    #subtracts 
        neglist=[]
        other1=other.poly
        for i in other1:
            dummytuple=(-1*i[0],)+i[1:]
            neglist.append(dummytuple)
        return self.__add__(Poly(neglist))

    def __mul__(self, other):     #multiplies
        if type(other)==int or type(other)==float:
            mullist=[]
            self1=self.poly
            for i in self1:
                dummytuple=(float(other)*i[0],)+i[1:]
                mullist.append(dummytuple)
            return Poly(mullist)
        mullist=[]
        self1=self.poly
        other1=other.poly
        for i in self1:
            for j in other1:
                mulnum=i[0]*j[0]
                muldeg=tuple(array(i[1:])+array(j[1:]))
                multerm=(mulnum,)+muldeg
                mullist.append(multerm)
        return Poly(mullist)

    def monodiv(self, other):       
    """divides the polynomial by the leading term of the other"""
        if type(other)==type(Poly([(0,0,0)])):
            if len(other.poly)==1 and len(self.poly)==1:
                T = self.isdivisible(other)
                if T:
                    a, b = self.poly[0], other.poly[0] 
                    anscoef=float(a[0])/float(b[0])
                    ansdeg=array(a[1:])-array(b[1:])
                    return Poly([(anscoef,)+tuple(ansdeg)])

    def __div__(self, other):
    """divides: gives quotient and remainder"""
        if type(other)==int or type(other)==float:
            return self.__mul__(1.0/other)             
        s=len(other)   #no. of polynomials
        quotient=[Poly(self.zero)]*s
        remainder=Poly(self.zero)
        dummyself=Poly(self.poly)
        while dummyself.poly != self.zero:
            i=0
            divisionoccurred = False
            while i<s and divisionoccurred == False:
                T = dummyself.isdivisible(other[i])
                if T:
                    quotient[i]=quotient[i]+
                    (dummyself.leadingterm().
                    monodiv(other[i].leadingterm()))
                    dummyself=dummyself-(dummyself.leadingterm().
                    monodiv(other[i].leadingterm()))*other[i]
                    divisionoccurred=True
                else:
                    i=i+1
            if divisionoccurred==False:
                remainder=remainder+dummyself.leadingterm()
                dummyself=dummyself-dummyself.leadingterm()
        return [quotient, remainder]

    def LCM(self, other):  #gives LCM
        lead_self=self.LT()
        lead_other=other.LT()
        LCM_term=[1]
        for i in range(len(lead_self)-1):
            LCM_term.append(max(lead_self[i+1], lead_other[i+1]))
        LCM=[tuple(LCM_term)]
        return Poly(LCM)

    def s_poly(self, other):   #gives S_Polynomial
        LCM = self.LCM(other)
        first = LCM.monodiv(self.leadingterm())
        second = LCM.monodiv(other.leadingterm())
        return first*self-second*other

    def iszero(self):   #checks if the polynomial is zero polynomial
        return self.poly==self.zero
        
            


"""GB of RNC starts"""
#n is no. of variables-1; varables are x0, x1,...,x_n

def encode_order(term, order=0):     #term is only tuple
"""order is induced in the polynomial:
  eg: x^3*y^4*z with y>z>x    is same as x^4*y*z^3       with x>y>z  """
    if order==0:
        return term
    term_list=list(term)
    for i in range(len(order)):
        term_list[i+1]=term[order[i]+1]
    return tuple(term_list)

def minor_helper(n, m, order=0): #gives 2minor object
""" helping function for generating all 2 X 2 minors of the matrix
     A = | x_0    x_1    ...   x_(n-1) |
             | x_1    x_2   ....   x_n        |
             
             """
             
             
    if n<=1:
        return "Not Possible"
    if n == 2:
        term_1 = [0]*(m+2)
        term_2 = [0]*(m+2)
        term_1[0], term_1[1], term_1[3] = 1, 1, 1
        term_2[0], term_2[2] = -1, 2
        poly = [encode_order(tuple(term_1), order),
         encode_order(tuple(term_2), order)]
        return [Poly(poly)]
    minorr=minor_helper(n-1, m, order)
    for i in range(n-2):
        term_1 = [0]*(m+2)
        term_2 = [0]*(m+2)
        term_1[0], term_1[i+1], term_1[n+1] = 1, 1, 1
        term_2[0], term_2[i+2], term_2[n] = -1, 1, 1
        poly = [encode_order(tuple(term_1), order), 
        encode_order(tuple(term_2), order)]
        minorr.append(Poly(poly))
    final_poly_1 = [0]*(m+2)
    final_poly_2 = [0]*(m+2)
    final_poly_1[0], final_poly_1[n-1], final_poly_1[n+1] = 1, 1, 1
    final_poly_2[0], final_poly_2[n] = -1, 2
    final_poly = [encode_order(tuple(final_poly_1), order), 
    encode_order(tuple(final_poly_2), order)]
    minorr.append(Poly(final_poly))
    return minorr

def minor(n, order=0):
""" Generates all 2 X 2 minors of the matrix
     A = | x_0    x_1    ...   x_(n-1) |
             | x_1    x_2   ....   x_n        |
             
             """
    
    return minor_helper(n, n, order)

def all_s_poly(n, order=0):  
""" Generates all possible S-Polynomials of 2 X 2 minors generated from matrix
     A = | x_0    x_1    ...   x_(n-1) |
             | x_1    x_2   ....   x_n        |
             
             """

    s_poly_list=[]
    minorr=minor(n, order)
    i=0
    while i<n*(n-1)/2-1:
        j=i+1
        while j<n*(n-1)/2:
            s_poly_list.append(minorr[i].s_poly(minorr[j]))
            j=j+1
        i=i+1
    return s_poly_list

def isgrobner(n, order=0):
"""checks if each of the S-polynomial is divisible by G 
    return all(map(lambda x: (x/minor(n, order))[1].iszero(), 
    all_s_poly(n, order)))

############################################################################
\end{verbatim}

%\subsection*{B}
%Following is the result of checking if $G_{n}$ is grobner basis or not for all possible monomial orders for $2\leq n \leq 7$. This result we used in proving Theorem 3.
%\\

\end{document}